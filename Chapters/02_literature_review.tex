%
% \chapter{Literature Review}
% In this chapter, I will introduce and discuss literature relating to both
% analytics markets and online regression methods with varying feature spaces.
%
% \section{Online Machine Learning}
%
% Online machine learning is the field of machine learning models that work not
% on a batch of data in one go, but instead receiving data in a streaming
% fashion. Online learning is usually based on an iterative updating of model
% parameters one or a few time steps at a time TODO:REF. While in batch learning
% we minimize some loss function over the dataset, online learning has the concept
% of regret minimization TODO:REF. Regret is the idea of how much lower the sum
% loss could have been if the model had chosen some other set of parameters from
% the beginning. TODO: I'M NOT SURE REGRET SHOULD BE INTRODUCED HERE, ITS TOO COMPLI
%
% An online version of linear regression is seen already in TODO:REF FIRST ONLINE
% LR. Online learning often focuses on hardware with limited capabilities, where
% training on an entire dataset at the same time would not fit in memory. Many
% tabular machine learning methods can be appropriated as online learning methods
% for this case, as seen in TODO: REF. Non-linear methods like decision trees can
% also be trained in an online fashion. This is e.g. seen in TODO:REF, where a
% version of online random forest models known as Mondrian forests is
% introduced.
%
% Online learning also allows for the model to adapt to changes in the underlying
% data, i.e. a non-stationary distribution. This can be done by exponential
% regretting algorithms TODO:REF
%
% \subsection{Online Learning with Varying Feature Spaces}
%
% Varying feature spaces present new problems for online learning mechanisms
% regarding how to deal with missing and newly appearing features. Several
% state-of-the-art online learning models that support varying feature spaces are
% presented in the survey article TODO:REF, where they name this subfield
% Utilitarian Online Learning.
%
% They present six models. Example models are: the OCDS model from TODO:REF which
% is a linear model and uses a linear mapping to reconstruct missing features by
% a generative graph approach; the OLVF model from TODO:REF which projects
% features into a shared space and uses slack and a hinge loss to only update
% once it has more information on new features TODO:CHECK IF TRUE; and the OVFM,
% which is a development of OCDS that attempts to improve how it deals with
% ordinal features.
%
% Utilitarian online learning has also been seen in semi-supervised settings in
% the OSLMF algorithm TODO: REF, where regression targets are only occasionally
% available. It is also a development on OCDS TODO:CHECK. Bayesian regression can
% also be performed in an online fashion by conditioning a Bayesian model on each
% incoming observation, as seen in TODO.
%
% \section{Analytics Markets}
%
% Analytics markets already exist theoretically in several forms, as mentioned.
% They belong to the study of information economics, which is a field that
% studies the effects of information on decision making, and how information
% behaves as a good in the market TODO:REF. Information as a good is classified
% most importantly by the fact that replicating it is essentially of no cost.
% Then, once one person acquires some information, they can freely replicate it.
% This complicates many microeconomic problems, like that of pricing a good,
% since regular goods are usually priced by how much one is willing to pay to
% gain exclusive access to it.
%
% \subsection{Revenue Distribution}
%
% Revenue distribution between sellers in analytics markets is most often done in
% literature using the Shapley value. However, one open problem in the literature
% is the problem of robustness to replication, first introduced in TODO: REF
% AGARWAHL. Robustness to replication is a new market property that relates to
% the outcome of a seller replicating their data and selling it several times in
% the market. Agarwahl et al. noted that in their setup, a seller could get a
% larger part of the revenue in a market by duplicating their own data, meaning
% that their setup is not robust to replication.
%
% Shapley values, however, exist in two variants, the observational and
% interventional Shapley values. Observational Shapley values are found by
% retraining the model with different available sets of features, while
% interventional Shapley values are found by leaving the model the same but
% setting features equal to their mean. The paper TODO: REF AGARWAHL uses
% observational Shapley values, but TODO:REF FALCO shows using the interventional
% Shapley value is actually robust to replication. However, the interventional
% Shapley value comes with more risk involved for feature sellers, especially if
% their features are highly correlated with other features present in the
% datasets.
%
% \subsection{Variants of Analytics Markets}
%
% Analytics markets have been seen in many variants in literature, especially
% with regards to which model is used by the market. TODO:REF PINSON proved that
% under some assumptions, any linear model minimizing a convex loss function can
% give rise to a fair market. TODO:REF 1 and 2 develop markets based on
% the LASSO regression and TODO. TODO:REF THOMAS develops a market based on a
% probabilistic forecasting method using Bayesian regression. TODO: REF PINSON
% also extended the market to the online setting with streaming features, using
% an exponentially smoothed Shapley value and splitting the market into an
% in-sample part and an out-of-sample part.
